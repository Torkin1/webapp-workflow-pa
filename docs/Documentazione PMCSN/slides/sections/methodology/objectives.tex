\begin{frame}{\subsecname (1/2)}
    \begin{enumerate}
        \item \textbf{Obiettivo 1}: implementare un modello in grado di rappresentare il sistema oggetto di studio e misurare:
        \begin{itemize}
            \item tempo di risposta,
            \item popolazione media,
            \item throughput medio.
        \end{itemize}
        Le metriche devono essere considerate sia localmente (per singolo nodo) che globalmente (su tutto il sistema).

        \item \textbf{Obiettivo 2}: valutare l’impatto di un sistema di autenticazione a due fattori per rendere i pagamenti più sicuri, considerando le stesse metriche dell'obiettivo precedente.
    \end{enumerate}
\end{frame}

\begin{frame}{\subsecname (2/2)}
    \begin{enumerate}
        \setcounter{enumi}{2}
        \item \textbf{Obiettivo 3}: osservare il comportamento del sistema in caso di carico di richieste maggiore:
        \begin{itemize}
            \item Incremento da 4320 job/h (1.2 job/s) a circa 5000 job/h (1.4 job/s).
        \end{itemize}
        Il confronto viene eseguito sulle stesse metriche dei primi due obiettivi.

        \item \textbf{Obiettivo 4}: migliorare il sistema individuando eventuali bottleneck per aumentarne le prestazioni.
    \end{enumerate}
\end{frame}