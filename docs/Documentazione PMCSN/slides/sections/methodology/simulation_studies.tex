\subsubsection{Tipologie di simulazione}
\begin{frame}{\subsecname: \subsubsecname}
    \begin{itemize}
        \item \textbf{Tipologia delle simulazioni}:
        \begin{itemize}
            \item \textbf{Simulazioni ad orizzonte infinito}: Batch-Means con:
            \begin{itemize}
                \item 100 batch
                \item 64 job per batch
            \end{itemize}
            \item \textbf{Simulazioni transitorie}:
            \begin{itemize}
                \item 5 run al variare del seed 
                \item 64 job per run
            \end{itemize}
        \end{itemize}
    \end{itemize}
\end{frame}

\subsubsection{Parametri di simulazione (1/2)}
\begin{frame}{\subsecname: \subsubsecname}
    \begin{enumerate}
        \item \textbf{Obiettivo 1}: Valutare il comportamento dei server nella configurazione base
        \begin{table}[H]
        \centering
        \begin{tabular}{cccc}
            \toprule
            \textbf{Server} & \textbf{Classe 1} & \textbf{Classe 2} & \textbf{Classe 3} \\
            \midrule
            \textbf{A} & 5 & 2.5 & 10 \\
            \textbf{B} & 1.25 & 0 & 0 \\
            \textbf{C} & 0 & 2.5 & 0 \\
            \bottomrule
        \end{tabular}
        \label{tab:service-rates-vanilla}
    \end{table}
    \vspace{0.2cm}
        \item \textbf{Obiettivo 2}: Analizzare l'impatto dell'\textbf{autenticazione a due fattori}
        \begin{itemize}
            \item Server A (classe 3): $6.6667~job/s$
            \item Server P: $1.4285~job/s$
            \item Server B: invariato.
        \end{itemize}
    \end{enumerate}
\end{frame}

\subsubsection{Parametri di simulazione (2/2)}
\begin{frame}{\subsecname: \subsubsecname}    
\begin{enumerate}
        \setcounter{enumi}{2}
        \item \textbf{Obiettivo 3}: Analizzare l'impatto di un nuovo \textbf{carico pesante}:
        \begin{itemize}
            \item Configurazione base per i tassi di servizio.
        \end{itemize}
        \item \textbf{Obiettivo 4}: Miglioramento delle performance
        \begin{itemize}
            \item Variazione del tasso di servizio del \textbf{server B}.
            \item Configurazione base per gli altri server.
        \end{itemize}
    \end{enumerate}
\end{frame}

\subsubsection{Grafici}
\begin{frame}{\subsecname: \subsubsecname}
Abbiamo analizzato tramite grafici:
    \begin{itemize}
        \item \textbf{Gli intervalli di confidenza (95\%)}:
        \[
        CI = \bar{x} \pm z \cdot \frac{\sigma}{\sqrt{n}}
        \]
        \begin{itemize}
            \item $\bar{x}$: media campionaria.
            \item $z$: valore critico della distribuzione normale standard ($z = 1.96$).
            \item $\sigma$: deviazione standard.
            \item $n$: numero di sample points.
        \end{itemize}
        \item \textbf{Distribuzione delle metriche}:
        \begin{itemize}
            \item Analisi della distribuzione per batch.
        \end{itemize}
    \end{itemize}
\end{frame}
