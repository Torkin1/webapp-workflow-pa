I valori di input, se non diversamente specificato, sono presi dal lavoro di \citet{DBLP:books/sp/Serazzi24}.

Le variabili indipendenti sono le seguenti:
\begin{itemize}
    \item Distribuzioni di probabilità dei servizi: Esponenziali;
    \item Distribuzioni di probabilità degli arrivi esterni: Esponenziali;
    \item Tempi di servizio medi per classe di job in ogni nodo: vedere \autoref{fig:average_service_per_job_class_by_node}
    \item Rate medi di arrivi esterni: valori che spaziano da $0.50$ a $1.20$ con un passo di $0.05 req/s$, per poi estendere fino a $1.40 req/s$ nel caso di carico pesante;
    \item Politiche di scheduling: PS per tutti i server coinvolti.
    \item Matrice di routing: consultando \autoref{fig:routing_matrix} e \autoref{fig:job_journey_with_classes} è possibile produrre la matrice di routing mostrata in \autoref{tab:routing-matrix}.
\end{itemize}

\begin{figure}
    \centering
    \includegraphics[width=1\linewidth]{figs/average_service_per_job_class_by_node.png}
    \caption{Tempi di servizio medi per classe di job in ogni nodo, nella versione vanilla (sinistra) e con 2FA (destra) \citep{DBLP:books/sp/Serazzi24}}
    \label{fig:average_service_per_job_class_by_node}
\end{figure}

\begin{figure}
    \centering
    \includegraphics[width=1\linewidth]{figs/routing_matrix.png}
    \caption{Class switch matrix \citep{DBLP:books/sp/Serazzi24}}.
    \label{fig:routing_matrix}
\end{figure}

\input{tables/matrix_routing}

Le variabili dipendenti sono le seguenti metriche di performance:
\begin{itemize}
    \item Tempo di risposta $T$: detto anche Tempo di Residenza, è l'intervallo di tempo che una richiesta trascorre all'interno di un nodo (o del Sistema) dal momento in cui entra fino al momento in cui esce;
    \item Popolazione $N$: numero di richieste all'interno di un nodo (o nel Sistema);
    \item Throughput $X$: numero di richieste soddisfatte sull'unità di tempo da un nodo (o dal Sistema). 
    \item Utilizzazione $\rho$: proporzione di tempo in cui il nodo è occupato dall'esecuzione delle richieste.
\end{itemize}