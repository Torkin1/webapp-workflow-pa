Procediamo con l'individuazione degli obiettivi della nostra analisi:
\begin{itemize}
    \item \textbf{Obiettivo 1}: ha lo scopo di implementare un modello in grado di rappresentare il sistema oggetto di studio e misurare tempo di risposta, popolazione e throughput medi. Tutte le metriche devono essere considerati sia locali, per singolo nodo, sia globalmente su tutto il sistema.
    \item \textbf{Obiettivo 2}: la WebApp decide di implementare un sistema di autenticazione a due fattori per rendere i pagamenti più sicuri. Ci siamo posti l'obiettivo di andare a valutare l'impatto di questa modifica sul sistema, valutando le stesse metriche dell'obiettivo precedente
    \item \textbf{Obiettivo 3}: vogliamo ora osservare il comportamento del sistema in caso di carico di richieste maggiore. Per i primi due obiettivi il massimo carico era di 4320 $job/h$ (ovvero 1.2 $job/s$) incrementato a circa 5000 $job/h$ (ovvero 1.4 $job/s$). Il confronto viene eseguito sulle stesse metriche dei primi due obiettivi.
    \item \textbf{Obiettivo 4}: l'ultimo obiettivo ha lo scopo di migliorare il sistema andando ad individuare eventuali bottleneck per migliorarne le prestazioni.
\end{itemize}